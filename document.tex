\documentclass[12pt,a4paper]{article}
\usepackage[left=2.00cm, right=2.00cm, top=2.00cm, bottom=2.00cm]{geometry}
\usepackage[tiny]{titlesec}
\usepackage[utf8x]{inputenc}
\usepackage[polish]{babel}
\usepackage[T1]{fontenc}
\usepackage{ucs}
\usepackage{amsmath}
\usepackage{amsfonts}
\usepackage{graphicx}
\usepackage{multicol}

\title{Pytania i odpowiedzi na obronę pracy inżynierskiej}
\author{
	Zachodniopomorski Uniwersytet Technologiczny\\
	Wydział Informatyki\\
	Szczecin
}
\date{\today}

\begin{document}
	\maketitle

	\section{Co to jest algorytm - cechy i właściwości}
	Algorytm jest skończonym, uporządkowanym ciągiem jasno zdefiniowanych czynności, koniecznych do wykonania postawionego  zadania.
	Cechy algorytmów:
	\begin{itemize}
		\item \textbf{poprawność} - algorytm daje oczekiwane wyniki,
		\item \textbf{jednoznaczność} - zawsze daje te same wyniki przy takich samych danych wejściowych,
		\item \textbf{skończoność} - wykonuje się w skończonej liczbie kroków,
		\item \textbf{sprawność} - czasowa - szybkość działania i pamięciowa.
	\end{itemize}
	Właściwości algorytmów:
	\begin{itemize}
		\item \textbf{efektywność} - algorytm powinien osiągać efekt końcowy możliwie niskim kosztem,
		\item \textbf{zgodność ze specyfikacją},
		\item \textbf{właściwość stopu} - algorytm powinien zatrzymać się w skończonym czasie (po wykonaniu lub mimo niewykonania postawionego zadania).
	\end{itemize}

	\section{Porównać pojęcia program, algorytm, procedura, funkcja, agent programowy.}
	Odpowiedź

	\section{Rodzaje zabezpieczeń systemów komputerowych}
	\begin{itemize}
		\item \textbf{fizyczne} - wszelkie zabezpieczenia przed otwarciem pokrywy komputera, zamki, blokady, zabezpieczenia antykradzieżowe, kontrola dostępu do obiektów i pomieszczeń, systemy przeciwpożarowe,
		\item \textbf{techniczne} - software, oprogramowanie antywirusowe, kontrola dostępu, szyfrowanie informacji
		\item \textbf{organizacyjne} - regulaminy dla osób korzystających z systemów informatycznych, polityka bezpieczeństwa,
		\item \textbf{personalne} - sprawdzanie pracowników dopuszczonych do danych o szczególnym znaczeniu, przestrzeganie odpowiednich procedur zwalniania i zatrudniania pracowników, szkolenia.
	\end{itemize}

	\section{Urządzenia wejścia i wyjścia}
	Odpowiedź

	\section{Scharakteryzować architekturę klient-serwer oraz klient-broker-serwer.}
	\begin{itemize}
		\item  \textbf{Klient-serwer} - program klienta (aktywny) wysyła żądania do serwera, który te zapytania przetwarza i dostarcza odpowiednią usługę. Z reguły serwer (pasywny) jest jeden i może obsługiwać wiele klientów.
		\item \textbf{Klient-broker-serwer} - pomiędzy klientem a serwerem jest pośrednik (broker), który jest odpowiedzialny za odbieranie wszystkich wiadomości, ich filtrowanie, określanie kto jest odbiorcą wiadomości oraz ich przesyłanie. Odpowiada również za przechowywanie danych o sesjach oraz autoryzację klientów.
	\end{itemize}

	\section{Wymienić i omówić metody wdrażania systemów informatycznych.}
	Odpowiedź

	\section{Scharakteryzować podstawowe modele baz danych.}
	\begin{itemize}
		\item  \textbf{Hierarchiczny} - w tym modelu przechowywane dane są zorganizowane w postaci drzewa. Informacja jest zawarta w dokumentach oraz w strukturze drzewa (podobnej do drzewa folderów na dysku komputera).
		\item  \textbf{Sieciowy} - uogólniony model hierarchiczny, rekordy mogą przyjmować strukturę dowolnego grafu.
		\item  \textbf{Relacyjny} - rekordy są grupowane w relacje (tabele). Dla każdej relacji musi zostać wybrany klucz główny, jednoznacznie identyfikujący dany rekord. Klucz obcy pozwala na powiązanie relacji między sobą (skrót myślowy). Większość relacyjnych baz danych korzysta z języka zapytań SQL. W modelu relacyjnym abstrahujemy od kolejności wierszy (rekordów) i kolumn (pól w rekordzie). Wiersz reprezentuje jeden rekord informacji np. osobę. Liczba kolumn jest z góry ustalona. Z każdą kolumną jest związana jej nazwa oraz dziedzina, określająca zbiór wartości, jakie mogą wystąpić w kolumnie.
		\item  \textbf{Obiektowy} - dane przyjmują postać obiektów. W przeciwieństwie do modelu relacyjnego, rekordy i relacje między nimi przechowywane są bezpośrednio (w formie obiektów, czyli struktur zwanych klasami), bez podziału na wiersze i kolumny.
	\end{itemize}

	\section{Czym wyróżnia się rozproszonych system informatycznych od innych.}
	Odpowiedź

	\section{Porównaj metody analizy obiektowej i strukturalnej w projektowaniu systemów informatycznych.}
	\begin{itemize}
		\item W podejściu \textbf{strukturalnym} dąży się do formalnej analizy systemu. W wyniku tej analizy tworzone są hierarchiczne struktury, których elementami są procesy, dane i związki zachodzące między nimi. Cechą charakterystyczną tego podejścia jest oddzielne modelowanie danych i procesów, wykorzystujące diagramowe i macierzowe metody i techniki.
		\item Podstawową różnicą między podejściem strukturalnym a \textbf{obiektowym} jest zintegrowane, jednoczesne modelowanie danych i procesów dziedziny przedmiotowej. System w podejściu obiektowym stanowi kolekcję różnych rodzajów, wzajemnie powiązanych elementów zwanych obiektami, spełniających w nim określoną rolę. Pojęcia klasy i obiektu umożliwiły powiązanie atrybutów (danych) i operacji (usług) w elementy, które łatwo przenieść koncepcyjnie na obiekty świata rzeczywistego.
	\end{itemize}

	\section{Scharakteryzować standardowy język zapytań do baz danych.}
	Odpowiedź

	\section{Na czym polega polimorfizm metod w programowaniu obiektowym i po co się go stosuje?}
	Polega na przedefiniowaniu metod klasy nadrzędnej w klasie pochodnej. Polimorfizm pozwala traktować różnorodne dane w ten sam sposób. Przykładowo mamy wiele klas (dziedziczących z jednej klasy abstrakcyjnej) dla różnych rodzajów figur geometrycznych a dla każdej z tych figur możemy policzyć jej pole. Polimorfizm pozwala nam w każdej z tych klas zaimplementować metodę wirtualną o takiej samej nazwie np. ObliczPole() i w zależności od typu obiektu w momencie wywołania metody zostanie wykonana ta prawidłowa. Pozwala to na odseparowanie implementacji od interfejsu, co z kolei ułatwia rozszerzanie funkcjonalności.

	\section{Wymienić i scharakteryzować metody testowania oprogramowania.}
	Odpowiedź

	\section{Wymienić metody ochrony danych w systemach baz danych.}
	\begin{multicols}{2}
		\begin{itemize}
			\item Kontrola dostępu,
			\item audyt wykonywanych operacji,
			\item uwierzytelnianie,
			\item szyfrowanie danych,
			\item kontrola integralności danych,
			\item kopie zapasowe,
			\item replikacja danych,
			\item mechanizm transakcji,
			\item ochrona w warstwie aplikacji.
		\end{itemize}
	\end{multicols}

	\section{Rola sterowników w dostępie do baz danych.}
	Odpowiedź

	\section{Zarządzanie procesami w systemach operacyjnych.}
	Odpowiedź

	\section{Co to jest system komputerowy, informacyjny, informatyczny.}
	Odpowiedź

	\section{Powody tworzenia systemów rozproszonych.}
	Odpowiedź

	\section{Środowiska programistyczne stosowane do obliczeń inżynierskich.}
	Odpowiedź

	\section{Rodzaje systemów operacyjnych (klasyfikacja i charakterystyka).}
	Odpowiedź

	\section{Podać klasyfikację języków programowania.}
	Odpowiedź

	\section{Paradygmaty programowania obiektowego.}
	Odpowiedź

	\section{Zadania systemu zarządzania bazami danych (DBMS).}
	Odpowiedź

	\section{Topologie sieci komputerowych.}
	Odpowiedź

	\section{Podstawowe składniki sprzętowe w sieciach komputerowych.}
	Odpowiedź

	\section{Zastosowania mikroprocesorów.}
	Odpowiedź

	\section{Metody kompresji danych.}
	Odpowiedź

	\section{Sprzętowe środki przyspieszania obliczeń.}
	Odpowiedź

	\section{Klasyfikacja usług internetowych.}
	Odpowiedź

	\section{Budowa procesora (CPU).}
	Odpowiedź

	\section{Technologie tworzenia stron internetowych.}
	Odpowiedź

	\section{Czym różnią się portal i wortal internetowy.}
	Odpowiedź

	\section{Przetwarzanie rozproszone – charakterystyka.}
	Odpowiedź

	\section{Przetwarzanie równoległe – charakterystyka.}
	Odpowiedź

	\section{Grafika rastrowa a grafika wektorowa.}
	Odpowiedź

	\section{Porównanie modeli odniesienia: ISO/OSI oraz TCP/IP.}
	Odpowiedź

	\section{Zadania warstwy transportowej.}
	Odpowiedź

	\section{Charakterystyka warstwy fizycznej.}
	Odpowiedź

	\section{Charakterystyka warstwy łącza danych.}
	Odpowiedź

	\section{Do czego służy protokół TCP, a do czego IP?}
	Odpowiedź

	\section{Rodzaje światłowodów - wady, zalety.}
	Odpowiedź

	\section{Scharakteryzować sieciowe systemy plików.}
	Odpowiedź

	\section{Wymień i opisz warstwy modelu OSI.}
	Odpowiedź

	\section{Podstawowe cechy standardów sieci bezprzewodowych WiFi.}
	Odpowiedź

	\section{Przedstawić budowę światłowodu.}
	Odpowiedź

	\section{Cechy charakterystyczne cyfrowych sieci ISDN.}
	Odpowiedź

	\section{Rodzaje i zastosowania macierzy dyskowych.}
	Odpowiedź

	\section{Zasada działania systemów klastrowych.}
	Odpowiedź

	\section{Zasada działania systemów ekspertowych.}
	Odpowiedź

	\section{Omów zasadę działania monitora (CRT lub LCD).}
	Odpowiedź

	\section{Wymienić i scharakteryzować rodzaje pamięci półprzewodnikowych}
	Odpowiedź

	\section{Przedstaw tablice prawdy AND, OR, XOR, zilustruj oznaczenie bramki, wymień przykładowe zastosowanie.}
	Odpowiedź

	\section{Wątki a procesy - na podstawie wybranego systemu. Wymienić wady, zalety.}
	Odpowiedź

	\section{Budowa typowego układu FPGA.}
	Odpowiedź

	\section{Podstawowe tryby adresowania systemów mikroprocesorowych}
	Odpowiedź

	\section{Hierarchia pamięci w systemie komputerowym, stronicowanie oraz koncepcja pamięci wirtualnej.}
	Odpowiedź

	\section{Omówić strukturę i funkcjonowanie systemu transmisyjnego.}
	Odpowiedź

	\section{Różnice między pamięcią statyczną i dynamiczną.}
	Odpowiedź

	\section{Problem synchronizacji przy transmisji danych i transmisja asynchroniczna}
	Odpowiedź

	\section{Uprawnienia plików na przykładzie systemu operacyjnego Unix/Linux.}
	Odpowiedź

	\section{Scharakteryzować sieciowe systemy plików.}
	Odpowiedź

	\section{Co to jest cykl życia oprogramowania i z jakich faz się składa?}
	Odpowiedź

	\section{Wymienić rodzaje diagramów w UML}
	Odpowiedź

	\section{Co oznaczają skróty ERD oraz DFD? Do czego się ich używa?}
	Odpowiedź

	\section{Przeciążanie funkcji i operatorów w języku C++.}
	Odpowiedź

	\section{Scharakteryzować instrukcje iteracyjne w przykładowym języku programowania}
	Odpowiedź

	\section{Omówić na czym polega przeciążanie funkcji i operatorów w języku C++.}
	Odpowiedź

	\section{Scharakteryzować mechanizmy dostępu do składowych klasy tworzonych statycznie i dynamicznie}
	Odpowiedź

	\section{Omów pojęcia agregacji i zawierania w diagramach UML.}
	Odpowiedź

	\section{Budowa i zasady działania wybranego urządzenia (drukarka laserowa, dysk twardy, pamięć USB, streamer, ect.)}
	Odpowiedź

	\section{Metody komunikacji człowiek-komputer.}
	Odpowiedź

	\section{Wymienić metody ekstrakcji wiedzy z danych.}
	Odpowiedź

	\section{Co to są drzewa decyzyjne i do czego służą?}
	Odpowiedź

	\section{Rekurencja i jej implementacja w językach wysokiego poziomu}
	Odpowiedź

	\section{Co to są algorytmy zachłanne – podać przykład takiego algorytmu.}
	Odpowiedź

	\section{Na czym polega haszowanie i gdzie ma ono zastosowanie?}
	Odpowiedź

	\section{Co to są problemy obliczeniowo trudne – podać przykład takiego problemu.}
	Odpowiedź

	\section{Maszynowa reprezentacja danych}
	Odpowiedź

	\section{Assembler, interpreter, kompilator – porównać i wyjaśnić pojęcia.}
	Odpowiedź

	\section{Zarządzanie pamięcią w Unix/Linux.}
	Odpowiedź

	\section{Zasady korzystanie z kluczy i pakietów kryptograficznych PGP (Pretty Good Privacy)}
	Odpowiedź

	\section{Metody reprezentacji wiedzy i wnioskowanie.}
	Odpowiedź

	\section{Zasady przetwarzanie transakcji w DBMS.}
	Odpowiedź

	\section{Narzędzia i środowiska wytwarzania oprogramowania.}
	Odpowiedź

	\section{Wzorce projektowe i programowe.}
	Odpowiedź

	\section{Metody podnoszenia niezawodności systemów wbudowanych.}
	Odpowiedź

	\section{Ryzyko i odpowiedzialność związana z systemami informatycznymi}
	Odpowiedź

	\section{Klasyfikacja systemów oprogramowania użytkowego.}
	Odpowiedź

	\section{Systemy wspomagające wytwarzanie oprogramowania – klasyfikacja, przykłady, funkcje.}
	Odpowiedź

	\section{Wymienić i scharakteryzować podstawowe techniki w grafice komputerowej.}
	Odpowiedź

	\section{Wymienić i scharakteryzować metody przetwarzania obrazów.}
	Odpowiedź

	\section{Zasady i metody tworzenia indeksów w bazach danych?}
	Odpowiedź

	\section{Rodzaje i sposób działania przerzutników.}
	Odpowiedź

	\section{Różnica pomiędzy automatem Mealy'ego a automatem Moore'a.}
	Odpowiedź

	\section{Różnica pomiędzy układami typu PLA a układami PAL.}
	Odpowiedź

	\section{Wymienić i omówić znanych światowych wynalazców w dziedzinie informatyki i telekomunikacji}
	Odpowiedź

	\section{Omówić sposoby prezentacji informacji oraz pojęcia informacji analogowej i cyfrowej, sygnału analogowego oraz cyfrowego.}
	Odpowiedź

	\section{Zdefiniować pojęcie widma sygnału oraz omówić numeryczne metody jego obliczania.}
	Odpowiedź

	\section{Omówić skalę decybelewą.}
	Odpowiedź

	\section{Co to jest szerokości pasma oraz przepływności kanału transmisyjnego.}
	Odpowiedź

	\section{Omówić zagadnienie modulacji, ze szczególnym uwzględnieniem modulacji cyfrowych}
	Odpowiedź

	\section{Wymienić znane media transmisyjne.}
	Odpowiedź

	\section{Omówić problem uwierzytelniania na przykładach: uwierzytelniania SYK, uwierzytelniania SYH, uwierzytelniania SYA oraz pojęcia hasła, karty magnetycznej, karta elektronicznej, karty identyfikacyjnej SIM oraz omówić techniki biometryczne.}
	Odpowiedź

	\section{Formaty danych liczbowych.}
	Odpowiedź

	\section{Omówić zasady wykonania operacji arytmetycznych w kodzie U2.}
	Odpowiedź

	\section{Omówić zasady wykonania operacji na liczbach zmiennopozycyjnych.}
	Odpowiedź

	\section{Różnice między pamięcią statyczną i dynamiczną.}
	Odpowiedź

	\section{Wymienić standardowe postacie wyrażeń boolowskich.}
	Odpowiedź

	\section{Omówić kombinacyjne i sekwencyjne układy logiczne.}
	Odpowiedź

	\section{Scharakteryzować poszczególne etapy procesu konwersji analogowo-cyfrowej.}
	Odpowiedź

	\section{Omówić ogólną charakterystykę filtrów w cyfrowych.}
	Odpowiedź

	\section{Opisać proces akwizycji i kodowania danych multimedialnych w kontekście zastosowania ich w systemach transmisji strumieniowej.}
	Odpowiedź

	\section{Wymienić i omówić podstawowe parametry stosowane przy definiowaniu jakości usług.}
	Odpowiedź

	\section{Wymienić i omówić podstawowe metody szeregowania pakietów.}
	Odpowiedź

	\section{Różnica między standardami JPEG i JPEG2000, rodzaje transformacji obrazu wykorzystywane w kodowaniu obrazów.}
	Odpowiedź

	\section{Scharakteryzować kod Graya jako przykład elementu wchodzącego w skład metod cyfrowej modulacji sygnału.}
	Odpowiedź

	\section{Różnica między kodami detekcyjnymi i korekcyjnymi - przykłady zastosowań.}
	Odpowiedź

\end{document}